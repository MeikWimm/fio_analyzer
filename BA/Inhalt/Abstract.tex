\section*{Zusammenfassung}
\label{sec:Zusammenfassung}

Mit Benchmarks kann man die Performanz von Hardware oder Software, wie etwa die Rechenleistung und die I/O Geschwindigkeit messen. 
Diese Messungen können stark oder schwach schwanken. Dies hängt unter anderem davon ab, ob die Tests gerade erst begonnen haben oder über einen längeren Zeitraum ausgeführt wurden.
Ein stationärer Zustand ist ein Muster, das sich mit der Zeit wiederholt und bei dem das Verhalten in einem Zeitraum von der gleichen Natur ist wie in jedem anderen Zeitraum.
Um zu ermitteln, wann der stationäre Zustand eingetreten ist, sind stabile Messwerte mit minimalen Schwankungen erforderlich.
Das Problem ist besonders bei der Netzwerkübertragung relevant, da zusätzlich die Netzwerkübertragung die Schwankung beeinflusst.
Diese Messungen der I/O-Geschwindigkeit werden mit dem Benchmark-Tool Fio durchgeführt, und es wird ein Programm entwickelt, das diesen Zustand analysiert. 

In dieser Arbeit wird die statistische Analyse durchgeführt, um zu ermitteln, ab wann sich Messzeiträume nicht mehr statistisch signifikant unterscheiden. Die Tests, die dafür verwendet werden, sind der t-Test, Tukey-HSD-Test und der Mann-Whitney-Test. 
Basierend auf der Analyse kann eine Aussage darüber getroffen werden, wie oft die fio-Messungen wiederholt werden müssen, um den stationären Zustand beim Festplattenbenchmarking zu erreichen.
Das Programm wird in Java implementiert.