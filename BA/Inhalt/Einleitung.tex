\chapter{Einleitung}
\label{cha:Einleitung}
1 Einf¨uhrung
Die Performanz eines Rechners ist die Leistungsf¨ahigkeit und Effizienz bei der
Ausf¨uhrung von Aufgaben [5]. Um herauszufinden welche Systeme die beste
Performanz liefert, sind Messungen notwendig, um sie vergleichen zu k¨onnen.
Solche Messungen und Vergleiche werden mit Benchmarks durchgef¨uhrt [14].
Mit ihnen wird die Rechenleistung oder auch Lese-/Schreibgeschwindigkeit der
Hardware oder auch Software mit standardisierten Tests gemessen [4]. Ein
Beispiel f¨ur getestete Software, sind virtuelle Maschinen wie die Java Virtual
Machine (JVM) [6] oder virtuelle Maschinen die reale Rechnern simulieren [9].
Durch Benchmarks wird die Performanz von virtuellen CPUs oder des Hyper-
visor analysiert. Auch f¨ur Cloud Computing Plattformen wird Benchmarking
betrieben, um auch Performanz zwischen verschieden Cloud-Plattformen ver-
gleichen zu k¨onnen. Diese Benchmarks verwenden z.B die Turnaround-Zeit
als Metrik, also die Zeit, die ein Prozess ben¨otigt, um vollst¨andig durchzu-
laufen [10]. Somit gibt es ein weites Spektrum an Auswahl von Systemen, wo
Benchmarking betrieben wird. Ein Beispielprogramm f¨ur so ein Benchmark Tool
ist das Fio1. Fio ist ein Tool, mit dem man die Bandbreitengeschwindigkeit vom
Lesen/Schreiben testet und sie als eine Textdatei (von Fio Logdatei genannt)
ausgeben kann. Diese Tests werden in Fio Jobs genannt.
Die I/O-Geschwindigkeit kann st¨arker oder schw¨acher vom tats¨achlich er-
reichbaren Maximalwert abweichen. Das Ziel des geplanten Tools ist es, den
station¨aren Zustand zu ermitteln. Ein Zustand, wo die Rechenleistung oder
Lese-/Schreibge-schwindigkeit sich nicht mehr beim Messen stark ¨andert und
sich einem konstanten Wert ann¨ahert, aber durch den Nichtdeterminismus nie
erreicht wird [3]. Um den station¨aren Zustand zu ermitteln, werden verschiedene
Methodiken aus der Varianzanalyse verwendet. Der Tukey-HSD-Test, t-Test
und Mann-Whitney-Test sind statistische Tests die im Analysetool genutzt wer-
den. Zusammen mit den Logs aus den Jobs des Fio’s wird die Standabweichung
sowie die Varianz berechnet. Dabei werden mit den Tests die statistische Sig-
nifikanz als auch den Zusammenhang zwischen den Jobs analysiert.
Im Expos´e werden zuerst die Grundlagen behandelt. Es werden das Fio-Tool
sowie die Definitionen des station¨aren Zustands und die verwendeten statistis-
chen Tests beschrieben. Im zweiten Teil wird das Konzept beschrieben, wie
das entwickelte Programm mit dem Fio-Tool arbeiten wird und es werden erste
Experimente durchgef¨uhrt. Im vorletzten und letzten Kapitel werden auf ver-
wandte Arbeiten eingegangen und es wird der Zeitplan beschrieben, wie viele
Wochen jeweils die Kapitel f¨ur die Bachelor Arbeit genutzt werden.
\section{Motivation}
\section{Ansatz}
\section{Aufbau der Arbeit}